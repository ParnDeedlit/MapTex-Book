\chapter{WebGL}

\section{材质Material}

标准的材质结构如下:
\begin{lstlisting}
    struct Material
    {
        vec3 ambient;    // 环境光照(Ambient Lighting)
        vec3 diffuse;    // 漫反射光照(Diffuse Lighting)
        vec3 specular;   // 镜面光照(Specular Lighting)
        float shininess; // 影响镜面高光的散射/半径
    };
\end{lstlisting}

Cesium的材质定义如下: 
\begin{lstlisting}
    // Source\Shaders\Builtin\Structs\material.glsl    
    struct czm_material
    {
        vec3 diffuse;    // 漫反射光照(Diffuse Lighting) vec3(0.0);
        float specular;  // 镜面光照(Specular Lighting) 0.0
        float shininess; // 影响镜面高光的散射/半径 1.0;
        vec3 normal;     // 法向量 materialInput.normalEC;
        vec3 emission;   // 发射光
        float alpha;     // 透明度
    };
    // Source\Shaders\Builtin\Structs\materialInput.glsl
    struct czm_materialInput  
    {
        float s;
        vec2 st;
        vec3 str;
        vec3 normalEC;
        mat3 tangentToEyeMatrix;
        vec3 positionToEyeEC;
        float height;
        float slope;
        float aspect;
        float uaspect;
    };
\end{lstlisting}    

Cesium的材质JavaScript对象定义如下: 
\begin{lstlisting}
// Source\Scene\Material.js
function initializeMaterial(options, result) {
    // 提取fabric中的type uniforms materials components信息
    var cachedMaterial = Material._materialCache.getMaterial(result.type);
    if (!defined(cachedMaterial)) {
        Material._materialCache.addMaterial(result.type, result);
    }

    createMethodDefinition(result); // 可以简单的认为就是按照不同的材质对struct czm_material的赋值操作
    createUniforms(result);
    createSubMaterials(result);
}
function createMethodDefinition(material) {
    var components = material._template.components;
    var source = material._template.source;
    if (defined(source)) {
        material.shaderSource += source + '\n'; // 直接使用glsl的语法
    } else {
        // 每种自定义的Material必须重写czm_getMaterial函数来实现各自的特定的参数传递
        // 其内部实现是通过解析fabric结构里面的components来自动生成对应的材质代码
        material.shaderSource += 'czm_material czm_getMaterial(czm_materialInput materialInput)\n{\n';
        // 获取从基类默认的材质结构体
        material.shaderSource += 'czm_material material = czm_getDefaultMaterial(materialInput);\n';
        if (defined(components)) {
            var isMultiMaterial = Object.keys(material._template.materials).length > 0;
            for (var component in components) {
                if (components.hasOwnProperty(component)) {
                    if (component === 'diffuse' || component === 'emission') {
                        // 针对散射光和发射光的特殊处理。如果可以复合使用不同的material就使用czm_gammaCorrect
                        // 来复合处理,否则进行常规参数的赋值处理
                        var isFusion = isMultiMaterial && isMaterialFused(components[component], material);
                        var componentSource = isFusion ? components[component] : 'czm_gammaCorrect(' + components[component] + ')';
                        material.shaderSource += 'material.' + component + ' = ' + componentSource + '; \n';
                    } else if (component === 'alpha') {
                        // 针对透明度的特殊处理,从代码分析和下面的常规参数感觉没有什么区别
                        material.shaderSource += 'material.alpha = ' + components.alpha + '; \n';
                    } else {
                        // 常规参数的赋值处理
                        material.shaderSource += 'material.' + component + ' = ' + components[component] + ';\n';
                    }
                }
            }
        }
        material.shaderSource += 'return material;\n}\n';
    }
}
function createUniform(material, uniformId) {
    // uniformType类型可能为 float、bool、channels、samplerCube、sampler2D、mat[2|3|4]、vec[2|3|4]
    var uniformType = getUniformType(uniformValue);
    // 本质上将对象化的JavaScript对象解析成glsl需要的格式
    // 并使用material._uniforms[newUniformId]来保存不同的材质的具体参数
    var newUniformId = uniformId + '_' + material._count++;
    if (uniformType === 'sampler2D') {}
    else if (uniformType === 'samplerCube') {}
    else if (uniformType.indexOf('mat') !== -1) {}
    else {}
}
\end{lstlisting}   

\section{Color举例}
\begin{lstlisting}
Material.ColorType = 'Color';
Material._materialCache.addMaterial(Material.ColorType, {
    fabric: {
        type: Material.ColorType,
        uniforms: {
            color: new Color(1.0, 0.0, 0.0, 0.5)
        },
        components: {
            diffuse: 'color.rgb',
            alpha: 'color.a'
        }
    },
    translucent: function (material) {
        return material.uniforms.color.alpha < 1.0;
    }
});
\end{lstlisting}  